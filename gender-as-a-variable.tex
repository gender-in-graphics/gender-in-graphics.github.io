\PassOptionsToPackage{table}{xcolor}
\documentclass[nonacm,sigconf,review,balance=false]{acmart}
\setcopyright{none}
\citestyle{acmauthoryear}
\setcitestyle{square}

\setcopyright{none}
\acmConference[Under Review]{Under Review}{Under Review}{Online} \acmYear{2021}
\acmDOI{nnn.nnn}

\settopmatter{authorsperrow=4}
%%%%%%%%%%%%%%%%%%%%%%%%%%%%%%%%%%%%%%%%%%%%%%%%%%%%%%%%%%%%%%%%%%%%%%%%%%%%%%
% UNICODE: make θ
\usepackage[mathletters]{ucs}
\usepackage[utf8x]{inputenc}
%%%%%%%%%%%%%%%%%%%%%%%%%%%%%%%%%%%%%%%%%%%%%%%%%%%%%%%%%%%%%%%%%%%%%%%%%%%%%%

%%%%%%%%%%%%%%%%%%%%%%%%%%%%%%%%%%%%%%%%%%%%%%%%%%%%%%%%%%%%%%%%%%%%%%%%%%%%%%
% Referencing
%%%%%%%%%%%%%%%%%%%%%%%%%%%%%%%%%%%%%%%%%%%%%%%%%%%%%%%%%%%%%%%%%%%%%%%%%%%%%%
\newcommand{\refequ}[1]{Eq.~(\ref{equ:#1})}
\newcommand{\refapp}[1]{Appendix \ref{app:#1}}
\newcommand{\reffig}[1]{Figure~\ref{fig:#1}}
\newcommand{\refsec}[1]{Section~\ref{sec:#1}}
\newcommand{\silvia}[1]{{\bf\textcolor[rgb]{0.9,0.6,0.6}{Silvia: #1}}}
%%%%%%%%%%%%%%%%%%%%%%%%%%%%%%%%%%%%%%%%%%%%%%%%%%%%%%%%%%%%%%%%%%%%%%%%%%%%%%
% Characters
%%%%%%%%%%%%%%%%%%%%%%%%%%%%%%%%%%%%%%%%%%%%%%%%%%%%%%%%%%%%%%%%%%%%%%%%%%%%%%
% "Remove" any exist single-letter commands
\providecommand{\A}{}
\providecommand{\B}{}
\providecommand{\C}{}
\providecommand{\D}{}
\providecommand{\E}{}
\providecommand{\F}{}
\providecommand{\G}{}
\providecommand{\H}{}
\providecommand{\I}{}
\providecommand{\J}{}
\providecommand{\K}{}
\providecommand{\L}{}
\providecommand{\M}{}
\providecommand{\N}{}
\providecommand{\O}{}
\providecommand{\P}{}
\providecommand{\Q}{}
\providecommand{\R}{}
\providecommand{\S}{}
\providecommand{\T}{}
\providecommand{\U}{}
\providecommand{\V}{}
\providecommand{\W}{}
\providecommand{\X}{}
\providecommand{\Y}{}
\providecommand{\Z}{}
\providecommand{\a}{}
\providecommand{\b}{}
\providecommand{\c}{}
\providecommand{\d}{}
\providecommand{\e}{}
\providecommand{\f}{}
\providecommand{\g}{}
\providecommand{\h}{}
\providecommand{\i}{}
\providecommand{\j}{}
\providecommand{\k}{}
\providecommand{\l}{}
\providecommand{\m}{}
\providecommand{\n}{}
\providecommand{\o}{}
\providecommand{\p}{}
\providecommand{\q}{}
\providecommand{\r}{}
\providecommand{\s}{}
\providecommand{\t}{}
\providecommand{\u}{}
\providecommand{\v}{}
\providecommand{\w}{}
\providecommand{\x}{}
\providecommand{\y}{}
\providecommand{\z}{}
% Now can just use renewcommand

%%%%%%%%%%%%%%%%%%%%%%%%%%%%%%%%%%%%%%%%%%%%%%%%%%%%%%%%%%%%%%%%%%%%%%%%%%%%%%
% Tables
%%%%%%%%%%%%%%%%%%%%%%%%%%%%%%%%%%%%%%%%%%%%%%%%%%%%%%%%%%%%%%%%%%%%%%%%%%%%%%
% \begin{table}
%   \centering
%   \arraystretch{1.2}
%   \setlength{\tabcolsep}{5.5pt}
%   \rowcolors{2}{white}{lightbluishgrey}
%   \begin{tabular}{l r r r r r r r r r r r r r r r r r r}
%     \rowcolor{white}
%       col1 & \makecell{long name \\ col2} &  \\
%     \midrule
%       item1 & entry1\\
%     \bottomrule
%   \end{tabular}
%   \caption{ }
% \end{table}
\usepackage{tabularx}
\usepackage{booktabs}
\usepackage{makecell}
\usepackage[table]{xcolor}
\definecolor{white}{rgb}{1,1,1}
\definecolor{lightbluishgrey}{rgb}{0.76471,0.84824,0.91647}

%%%%%%%%%%%%%%%%%%%%%%%%%%%%%%%%%%%%%%%%%%%%%%%%%%%%%%%%%%%%%%%%%%%%%%%%%%%%%%
% Figures
%%%%%%%%%%%%%%%%%%%%%%%%%%%%%%%%%%%%%%%%%%%%%%%%%%%%%%%%%%%%%%%%%%%%%%%%%%%%%%
\usepackage{subcaption}
\usepackage{wrapfig}
\usepackage{graphicx}

\usepackage{listings}
%%%%%%%%%%%%%%%%%%%%%%%%%%%%%%%%%%%%%%%%%%%%%%%%%%%%%%%%%%%%%%%%%%%%%%%%%%%%%%
% Line widths and font sizes
%%%%%%%%%%%%%%%%%%%%%%%%%%%%%%%%%%%%%%%%%%%%%%%%%%%%%%%%%%%%%%%%%%%%%%%%%%%%%%
\usepackage{layouts}
\makeatletter 
% \layoutdetails will insert a table listing the layout sizes and fonts in use
\newcommand{\layoutdetails}{%
\begin{tabular}{ll}
 \texttt{\textbackslash{textwidth}} & \printinunitsof{in}\prntlen{\textwidth} \\
\texttt{\textbackslash{linewidth}} & \printinunitsof{in}\prntlen{\linewidth} \\
Main text font &  \f@size pt \f@family \\
\sffamily \small Caption text font &  \sffamily \small \f@size pt \f@family \\
\end{tabular}%
}
\makeatother 


\begin{document}

\title{Gender and Sex in the Computer Graphics research literature}

\author{Ana Dodik}\authornote{Joint First Authors}
\affiliation{\institution{Meta Platforms}}

\author{Silvia Sellán}\authornotemark[1]
\affiliation{\institution{University of Toronto}}

\author{Theodore Kim}
\affiliation{\institution{Yale University}}

\author{Amanda Phillips}
\affiliation{\institution{Georgetown University}}



\begin{abstract}
    We survey the treatment of sex and gender in the Computer Graphics research
    literature from an algorithmic fairness perspective. We conclude
    established practices on the use of gender and sex in our research community are scientifically incorrect and constitute a
    form of algorithmic bias with potential harmful effects. We propose ways of addressing these trends as technical limitations.
\end{abstract}


\maketitle

%\ana{I understand wanting to focus on SIGGRAPH, but I think we should consider broadening the scope to include all of visual computing research. Computer vision is \textit{especially} guilty of this.}
\vspace{-0.1cm}
\section{Introduction}

\ted{this is a sample Ted Kim comment}

References to sex and gender can be found all throughout the Computer Graphics research literature: a dataset is said to contain images of men and women, user study participants are reported to have a certain male/female ratio, a body modeling algorithm trains two different gendered models, a voice modification method is said to work on male and female voices, etc.

% also mention society, maybe industry as fields moving away from this binary

% Amanda comments:

% flip-flop between sex and gender: commit to one or other or both. (So far committed to both)

% be clear the critique is about binarism, not the use of gender itself

% Claire Hemmings: cite journals and not person

% more specific examples in the text

The scientific consensus around the concepts of sex and gender has greatly evolved in the past decades (see, e.g., \cite{pmid30377332}). As surveyed by \citet{fausto2012sex}, \emph{sex} is not one but a combination of many biological classifications (\emph{chromosomal sex, hormonal sex, reproductive sex, ...}) which cannot be unambiguously assigned in a binary way to as much as one in 50 people \cite{blackless2000sexually}.
\emph{Gender}, on the other hand, is used to refer to an individual's self-identity \cite{money1972man}, their performance of certain acts as shaped by social expectations \cite{butler2003gender} and arbitrary organizational structures that segregate people in different public bathrooms and even decide who can access  education \cite{lorber1994paradoxes}. By all these contemporary definitions, gender is non-binary, fluid and culturally-specific. Furthermore, assuming outdated binary definitions of sex and gender is not just scientifically incorrect, but is also harmful (see \cite{un2015report}) to those who conform the least to this artificial binary (e.g, \emph{intersex}, \emph{transgender}, \emph{non-binary} people), to whom we will onwards refer to as \emph{gender non-conforming} individuals.

Disappointingly, we observe that the treatment of sex and gender in SIGGRAPH Technical Papers
still answers to a traditional binary understanding of it that excludes intersex and many transgender and
gender non-conforming people. In what follows, we will use an algorithmic fairness perspective to argue that this introduces forms of algorithmic bias that harm our scientific integrity.
We will examine the harmful real-world consequences of the algorithmic bias introduced by our modeling
choices with respect to gender on how gender non-conforming people interact with
our technology in their daily lives. We will advocate for reexamining our
treatment of gender and show that this will not only correct problematic practices established in
our community, but also open the door to whole new avenues of research.

\vspace{-0.1cm}
\section{Survey}

%\ana{So a small suggestion on how to make this more formal, we could indicate which type of algorithmic unfairness the different papers introduce, e.g.~\cite{FriedmanAndNissenbaum}.}

Inspired by the work of \citet{keyes2018misgendering}, we conducted a survey of
all technical papers presented at SIGGRAPH North America and SIGGRAPH Asia since
2015 (see \emph{Supplemental Material}). We observed references to sex and gender routinely
throughout, varying in nature from demographic information reported about user
study participants to gender-specific algorithms. Whenever gender or sex is used
explicitly as a variable, it is always as a binary one. Sex and gender are never given a precise definition in all the reviewed
literature, and appear to be used implicitly as a proxy for anything from body
proportions to facial expression to patterns in speech.

An analysis of the above from an algorithmic fairness perspective (\refsec{analysis}) reveals a worrying status quo in the use of gender and sex
as variables in Computer Graphics. As we
mention examples of works that perpetuate these trends, we stress that we do not
associate any malicious intent to any. Rather, we wish to show how seemingly
neutral, well-established practices in our community can lead to us unwittingly
perpetuating forms of algorithmic bias.

%\ana{There is something to be said about the entire ``It's just the data thats biased'' discussion we keep having in ML ad nauseam. In our examples, it's not just the data that's the root of algorithmic unfairness, it's the active decisions of algorithm designers. For practitioners this means that, even if you don't have access to, e.g.~body scans of non-binary people, you can still try to remove a part of that bias that is due to algorithm design decisions.}
\vspace{-0.1cm}
\section{Algorithmic Fairness Analysis}\label{sec:analysis}

Our survey shows that the current use of gender and sex in the Computer Graphics literature is at best ill-defined, and at worst incorrect.
In this section,
we apply the framework of \citet{Suresh2021}, which categorizes types of bias according to the stages of a system's lifecycle 
(see also \cite{fairnesssurvey,FriedmanAndNissenbaum,olteanu2019social})
We give examples of how different types of bias occur in our surveyed work and show that these are \emph{technical} limitations that impede our community's goal of producing precise, high-quality and reproducible research.

\paragraphtight{Representation bias} A part of a population may be poorly represented by a dataset, for example, because the sampling procedure is biased not to include people of non-binary genders (\emph{sample selection bias}) or because no care is taken to ensure algorithms perform equally well in  groups of \emph{underrepresented} sex or gender. Despite the prevalence of these individuals in the general population, we could not identified a single paper (O3) that explicitly mentioned them as part of datasets (\dataset) or user study participants (\userstudy). The sampling procedure may have been accidentally designed to exclude these individuals, or it might be due to measurement bias. We did not identify any work that explicitly analyzed any type of representation bias experienced by gender non-conforming individuals (\binary).
% \begin{itemize}
%     \item \emph{Sample selection bias} occurs when the sampling procedure is biased not include non-binary people. Given the scale of the works reviewed and the reported statistics about the prevalence of such individuals, the works we reviewed should include non-binary individuals; however, we could not identify a single paper that explicitly mentioned them as part of datasets or user study participants. The sampling procedure may have been accidentally designed to exclude these individuals, or it might also be due to measurement bias.
%     \item A dataset can lead to algorithmic unfairness if it is uniformly sampled, but \emph{contains under-represented groups}. A uniformly sampled dataset with $1000$ people is expected include between $1$ and $20$ intersex or gender non-conforming individuals. This means that the algorithms trained on this dataset are more likely to produce worse results on these individuals than in the general population. In our survey, we did not identify a single paper that acknowledged the existence of intersex and gender non-conforming people; consequently, we did not identify a single paper that explicitly attempted to correct this type of representation bias.
% \end{itemize}

\paragraphtight{Historical bias} Data, despite being abundant and perfectly sampled, may encode existing prejudice. For example, a \emph{gender classifier} (\classifier) trained on portrait image data collected in a society where social norms dictate gender expression might learn that ``wearing a dress'' means woman, and ``short hair'' means man.

\paragraphtight{Measurement bias} Bias may be introduced through the selection and measurement of features and target variables.
We observed that many works use sex or gender as imprecise \emph{proxies} (\var) for attributes like \emph{commonly co-occuring bodily characteristics} or \emph{speech characteristics}, where it is possible that the authors would be better served using other less abstract features (e.g., hair length, or voice pitch). We even observed works that combine several of these proxies into one; for example,
conversational agents that use gender to refer to both voice pitch \emph{and} culturally acquired speech inflections.
%virtual try-on algorithms that use gender both as a proxy for body geometry \emph{and} cultural choices in attire.

Furthermore, when gender or sex was chosen as a feature or target variable, it was always (\binary) through an \emph{inaccurate method of measurement} (treating gender as a binary variable, non-binary individuals cannot be captured by design even if they are in the dataset) and often (\classifier) through an \emph{incorrect method of measurement} (using image-based gender classifiers as opposed to self-identification, gender non-conforming individuals may be misidentified).

\paragraphtight{Omitted variable bias} The success of using a certain feature may be overemphasized if it correlates with another important feature that has been omitted from the model (see e.g., \cite{clarke2005phantom}). For example, \emph{gender} or \emph{sex} are likely not as discriminative as variables when the result is also conditioned on \emph{hair length}, \emph{hip width} or \emph{mean voice frequency}. In our survey, where gender or sex's use was justified because of an assumed improvement in model accuracy (\var), we found no effort to indentify if the success was due to omitted variables.

% During our literature review, we identified the presence of the following issues:
% \begin{itemize}
%     \item The frequent usage of \emph{proxies}, both as features and as target variables. For example, we observed works that use \emph{sex} or \emph{gender} to mean \emph{comonly co-occuring bodily characteristics} or \emph{a set of voice attributes}.
%     The use of \emph{sex} and \emph{gender} as proxy variables is not only harmful, but often also imprecise and inaccurate from a technical stand-point. It is quite possible that the authors would be better served using other less abstract features (e.g. hair length, or voice pitch). The imprecision can compound further when a single work combines several understandings of sex and gender; for example, we observed virtual try-on algorithms use gender both as a proxy for body parameters \emph{and} cultural choices in attire, and conversational agents use it to denote both the pitch of the agent's voice as well as cultural traits in their verbal and non-verbal communication. The usage of these variables can further lead to \emph{omitted variable bias}, which we discuss later.
%     \item An \emph{inaccurate} method of measurement. For example, all of the papers we reviewed treat gender as a \emph{discrete binary variable}. Therefore, even if data was collected from gender-diverse participants, the method of measurement would not be able to capture that.
%     \item An \emph{incorrect} method of measurement. We identified works that both propose and use gender labeling of image, voice and body geometry data, in many cases by automated systems or manual third parties as opposed to participant self-identification. Similarly, we observed works that report many user study participants' gender as "unknown", which may mean it is the researchers themselves who are attempting to assume their participant's gender without asking them to self-identify it. Algorithmic bias occurs in this situation, even if special care is taken to collect data of gender non-conforming individuals.
% \end{itemize}.



%\paragraphtight{Aggregation bias}\silvia{to-do} \ana{I need help on this one, since I cannot think of any examples. From the survey paper: ``Aggregation bias (or ecological fallacy) arises when false conclusions are drawn about individuals from observing the entire population. An example of this type of bias can be seen in clinical aid tools. Consider diabetes patients who have apparent morbidity differences across ethnicities and genders.
%Specifically, HbA1c levels, that are widely used to diagnose and monitor diabetes, differ in complex ways across genders and ethnicities. Therefore, a model that ignores individual differences will likely not be well-suited for all ethnic and gender groups in the population.
% This is true even when they are represented equally in the training data. Any general assumptions about subgroups within the population can result in aggregation bias.''}

%\paragraphtight{Learning bias} \silvia{to-do}Algorithm designers will often unknowingly introduce bias into their methods. For example, using a regularizer that makes sure to regress to an ``average'' body is likely to make the algorithm worse for people far away from the average. Alternatively, optimizing for a model's accuracy might lead to the statistical pairity decreasing accross different groups.

\paragraphtight{Evaluation bias} These are biases introduce during the evaluation of an algorithm. For example, we observed works in body modeling that provide binary segregated parametric models (O3). We observed these being used to evaluate \emph{other} works with orthogonal contributions, like virtual try-on or motion capture. If our research community settles on benchmarks with biased data, the development of models that conform to those biases is encouraged.

%\silvia{to-do}Evaluation bias encompases all bias that is introduced during the evaluation of an algorithm. If the computer graphics community settles on benchmarks with biased data or metrics, it creates a knock-on effect where the development and deployment of models that performed well on those benchmarks is further encouraged. This is one of the reasons it is particularly dangerous to "leave the fairness questions" for future work.

% \subsection{Real world harm}

% Unlike our survey, the effects of algorithmic bias do not stop at the publication stage. Rather, harm is introduced as a consequence of a model being used in the real world, known as \emph{deployment bias}.



\paragraphtight{Deployment bias} The effects of algorithmic bias do not stop at the publication stage; rather, harm is introduced as a consequence of a model being deployed or published in the real world. For example, the publication exclusively of papers with a binary understanding of sex and gender incentivizes researchers (and reviewers) to conform to that definition (\binary).
This can also lead to \emph{feedback loops}: if, as we have observed, gender non-conforming people are not included in a virtual clothing try-on system, they are less likely to use them, causing the data about the performance of such a system to be skewed to include fewer gender non-conforming people. Finally, a biased system can cause further bias by nudging users to artificially change their behaviour: for example, a trans person might feel the need to change the pitch of their voice in order to not get misgendered by an algorithm, thus also skewing the collected data.

\vspace{-0.1cm}
\subsection{Real world harm}

The discussed technical limitations of the reviewed algorithms are not only academic in nature, but can and do lead to real world harm. As Computer Graphics is being increasingly applied to other fields, such as for processing geometric data in medicine, or for synthetic dataset generation in computer vision with numerous downstream applications \cite{cars, chen2021synthetic, dhs}, it is paramount to understand that our algorithms can and will be used in novel ways potentially causing harm in ways we did not intend.

The algorithmic fairness literature disambiguates between representational and allocative harms \cite{barocas-hardt-narayanan}.

\paragraphtight{Representational harms} These encompas the perpetuation of harmful stereotypes or cultural norms that subject individuals to denigration. For example, it is well documented that airport body scanners routinely subject gender non-conforming passangers to public humilliation \cite{tsa}.

\paragraphtight{Allocative harms} These are caused when certain groups are denied access to an opportunity or a resource because of algorithmic bias. For example, a virtual try-on experience based on the algorithms in our survey might accidentally exclude precisely the people with non-normative bodies who are most in danger in traditional physical changing rooms \cite{changingroom}.

Finally, ignoring the existence of gender non-conforming individuals in our research (O3) creates an alienating and exclusionary environment for gender non-conforming members of our very own research community, going directly against SIGGRAPH's goal to be \emph{a model of inclusion, equity, access and diversity for all}.

\vspace{-0.1cm}
\section{Where do we go from here?}

Our analysis reveals that the common use of sex and gender in the Computer Graphics literature can pepper our research with algorithmic bias. Our disambiguated study shows bias is introduced all through the modeling process; thus, we argue algorithmic fairness cannot be an afterthought but rather present at all stages of our research. While we have focused on sex and gender, we hope our work contributes to the broader conversation about algorithmic fairness in our field, which includes racial biases \cite{kim2021countering}.

While different real-world constraints might not make it realistic for a research group to successfully mitigate certain sources of bias, the potentially introduced biases should at the very least be acknowledged. For example, none of the reviewed papers included any of the various algorithmic fairness metrics (for a summary, see \cite{fairnesssurvey, fairnessmetrics}) in their evaluation, nor did they even include a discussion of the potential harm their methods could be causing by their treatment of sex and gender.

Further, we argue that the issues raised by our survey are not only potentially harmful to under-represented populations, but also often \emph{scientific} limitations which are not well researched. If a method cannot model a class of humans by design, or if a production system fails for a subsection of the population, these are fundamental \emph{technical} limitations that should be discussed as such.

We believe gender and sex can have a place in our research. For example, it is probably advisable to report them among demographic statistics of datasets or user study participants (as long as they are self-reported and treated as non binary in agreement with the scientific consensus) to
safeguard against the “male default” plagueing our science. In most other cases we observed, however, we argue gender or sex could and should have been replaced by other, more accurate variables.

It bears mentioning that our observed entrenchment in sex and gender as relevant variables is a rare example of Computer Graphics research
lagging behind the needs of our partner industries. The latest
photorrealistic character modeller by \citet{metahuman} and the Cloud Vision API by \citet{googlegender} have removed all references to sex and gender. Video games as diverse as \emph{Animal Crossing} and \emph{Forza Horizon} completely decouple attributes like body proportions, voice pitch, hairstyle and prononouns from one another.

We acknowledge that our proposed break with tradition may bring with it effort
and difficult conversations, but these are challenges worth facing in the
interest of scientific advancement as well as producing a fairer, more inclusive
future.
% \begin{itemize}
%     \item Assuming an implicit definition of (e.g. binary) gender might incentivize future researchers to conform to that definition. This is additionally problematic when researchers from cultures with a different understanding of gender need to adjust to foreign cultural norms. (\ana{Amanda, I might need some help with this one.})
%     \item Deploying an algorithm can lead to \emph{feedback loops}. For example, if gender-diverse people have poor experiences with clothing size recommender systems, they are less likely to use them. As a consequence, the data about the performance of such a system will be skewed to include fewer gender-diverse people. Using such data to further optimize the systems can lead to compounding effects.
%     \item A deployed system can cause harm by nudging the users to artificially change their behavior. For example, a trans person might feel the need to change the pitch of their voice in order to not get misgendered by an algorithm.
%     \item Deploying a body model which has ``male'', ``female'', and ``gender neutral'' variants might lead to unintentional harm, if, for example, the ``gender neutral'' model is used only that if the algorithm does not have enough confidence the user is either ``male'' or ``female''. This can lead to gender trans or non-confirming individuals being actiely told that they are not recognized as the gender they identify with.
% \end{itemize}



%\ana{I don't know where to fit this: a proposed new conversational agent might learn a correlation between a person's visual appearance and certain traits they exhibit in non-verbal communication. Suggestions?}

%\silvia{to-do}

% In general, we have found the gender-related language to be imprecise, which hinders the clarity of the presentation. Oftentimes, we found that \emph{gender} and \emph{sex} are used seemingly interchangebly, and most of the times it is not even clear from context which one the authors intended to use. In fact, we argue that it is impossible to know a-priori if a trained model has picked up on \emph{gender} characteristics, and not on the characteristics of \emph{sex} or \emph{gender expression}, without looking at algorithmic fairness metrics accross different subgroups.

% society tends to conflate gender expression and gender identity.

% A worrying trend we noticed is that none of the reviewed papers provide an analysis of the algorithmic biases that they potentially introduce. While different real-world constraints might not make it realistic for a research group to successfully mitigate certain sources of bias, the potentially introduced biases should at the very least be acknowledged. For example, none of the reviewed papers included any of the various algorithmic fairness metrics (for a summaries, see \cite{fairnesssurvey, fairnessmetrics}) into their evaluation, nor did they even include a discussion of the potential harm their methods could be causing.

% Algorithmic fairness evaluations and discussions are left out of computer graphics papers not because they happen to be difficult or time-consuming, but because they are deemed unnecessary by the people who would at large be either unaffected or positively affected by the introduced biases.

% However, we argue that the problems introduced in these methods are not only potentially harmful to under-represented populations, but also often \emph{technically limited ways which are not well researched}. If a method cannot model a class of humans by design, or if a production system fails for a subsection of the population, these are fundamental \emph{technical} limitations. The question is then, why does our community prioritize solving these limitations disproportionatly less then other technical problems?


% \ana{Some suggestions for this section:
% \begin{itemize}
%     \item Algorithmic fairness should not be an afterthought, but something we plan for from the beginning, as it can be introduced in every step of algorithm development.
%     \item Data collection should account for how much representation is necessary from each marginalized group.
%     \item Algorithmic fairness metrics exist and are easy to implement. Use them, and report them in your research.
%     \item In our review, we found that gender or sex could have often been replaced by another variable and it would have been more accurate.
%     \item We argue that discussions around algorithmic fairness need to become front-and-center within our own community, instead of being relegated to other venues or "future work".
% \end{itemize}}

% \ana{This paragraph might be better suited for this section: It bears mentioning that our research community's entrenchment in the
% traditional gender binary is a rare example of Computer Graphics research
% lagging behind the needs of our partner industries. \emph{Metahuman}, the latest
% photorrealistic character modeller by \citet{metahuman} has no mention of
% gender; \citet{googlegender} removed all gender references from its Cloud Vision
% API; video games as diverse as \emph{Animal Crossing: New Horizons},and \emph{Forza Horizon 5} completely decouple attributes
% like hairstyle, body proportions, voice pitch and prononouns from one another.}

% \ana{This might also be better for the conclusion: As Computer Graphics researchers, we must consider our role in shaping whose stories get to be told and who gets to seem themselves represented in the entertainment culture.}

% We believe the reasons above to be enough to make us reevaluate the role of
% gender in our community's scientific literature.

% For example, the reporting of gender among other demographic information in user
% study participants and dataset collection subjects answer to a scientifically
% positive goal (experimental transparency) as well as an ethical one, to
% safeguard against the “male default” that plagues science and has plagued it
% since its infancy. However, we found instances in our survey of participants
% being reported as of “unknown gender”, which may indicate that their gender is
% being assumed post facto by researchers as opposed to self reported, leading to
% the potential misidentification and exclusion of gender non-conforming
% individuals or of those from certain ethnicities (see e.g.,
% \cite{santamaria2018comparison,buolamwini2018gender}). Therefore, we would argue
% it is still advisable to include this kind of data, as long as it is self
% reported by participants who are given a breadth of gender options not
% restricted to the traditional binary ones.

% On the other hand, the scientific and ethical harm caused by gender-segregated
% algorithms is likely too significant to offset any possible benefits. At the
% very least, these choices should be justified and their consequences in terms of
% excluding gender non-conforming individuals should be examined and clearly
% stated. Eventually, we hope that our field evolves to address these limitations
% and move beyond the outdated gender binary. We trust that our fellow researchers
% share our scientific excitement in this new frame of reference and the potential
% novel research directions it opens; for example:
% \begin{itemize}
%     \item What is a complete parametric model for the human body that is
%     decoupled from gender and accurately represents the diverse bodies of all
%     humans, regardless of whether they conform to traditional gender norms?
%     \item How can our research inform or contrast more modern understandings of
%     gender? Can data-based methods be used to evaluate cultural differences in
%     gender presentation?
%     \item How can we evaluate our algorithms for bias towards the gender binary?
%     What tools are needed to obtain or synthesize data that covers more diverse
%     experiences of gender?
% \end{itemize}

% We acknowledge that our proposed break with tradition may bring with it effort
% and difficult conversations, but these are challenges worth facing in the
% interest of scientific advancement as well as producing a fairer, more inclusive
% future.

\bibliographystyle{ACM-Reference-Format}
\bibliography{references.bib}

\end{document}
