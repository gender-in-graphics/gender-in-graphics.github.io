\PassOptionsToPackage{table}{xcolor}
\documentclass[acmtog,nonacm,review,balance=false]{acmart}
\citestyle{acmauthoryear}
\setcitestyle{square}

%%%%%%%%%%%%%%%%%%%%%%%%%%%%%%%%%%%%%%%%%%%%%%%%%%%%%%%%%%%%%%%%%%%%%%%%%%%%%%
% UNICODE: make θ
\usepackage[mathletters]{ucs}
\usepackage[utf8x]{inputenc}
%%%%%%%%%%%%%%%%%%%%%%%%%%%%%%%%%%%%%%%%%%%%%%%%%%%%%%%%%%%%%%%%%%%%%%%%%%%%%%

%%%%%%%%%%%%%%%%%%%%%%%%%%%%%%%%%%%%%%%%%%%%%%%%%%%%%%%%%%%%%%%%%%%%%%%%%%%%%%
% Referencing
%%%%%%%%%%%%%%%%%%%%%%%%%%%%%%%%%%%%%%%%%%%%%%%%%%%%%%%%%%%%%%%%%%%%%%%%%%%%%%
\newcommand{\refequ}[1]{Eq.~(\ref{equ:#1})}
\newcommand{\refapp}[1]{Appendix \ref{app:#1}}
\newcommand{\reffig}[1]{Figure~\ref{fig:#1}}
\newcommand{\refsec}[1]{Section~\ref{sec:#1}}
\newcommand{\silvia}[1]{{\bf\textcolor[rgb]{0.9,0.6,0.6}{Silvia: #1}}}
%%%%%%%%%%%%%%%%%%%%%%%%%%%%%%%%%%%%%%%%%%%%%%%%%%%%%%%%%%%%%%%%%%%%%%%%%%%%%%
% Characters
%%%%%%%%%%%%%%%%%%%%%%%%%%%%%%%%%%%%%%%%%%%%%%%%%%%%%%%%%%%%%%%%%%%%%%%%%%%%%%
% "Remove" any exist single-letter commands
\providecommand{\A}{}
\providecommand{\B}{}
\providecommand{\C}{}
\providecommand{\D}{}
\providecommand{\E}{}
\providecommand{\F}{}
\providecommand{\G}{}
\providecommand{\H}{}
\providecommand{\I}{}
\providecommand{\J}{}
\providecommand{\K}{}
\providecommand{\L}{}
\providecommand{\M}{}
\providecommand{\N}{}
\providecommand{\O}{}
\providecommand{\P}{}
\providecommand{\Q}{}
\providecommand{\R}{}
\providecommand{\S}{}
\providecommand{\T}{}
\providecommand{\U}{}
\providecommand{\V}{}
\providecommand{\W}{}
\providecommand{\X}{}
\providecommand{\Y}{}
\providecommand{\Z}{}
\providecommand{\a}{}
\providecommand{\b}{}
\providecommand{\c}{}
\providecommand{\d}{}
\providecommand{\e}{}
\providecommand{\f}{}
\providecommand{\g}{}
\providecommand{\h}{}
\providecommand{\i}{}
\providecommand{\j}{}
\providecommand{\k}{}
\providecommand{\l}{}
\providecommand{\m}{}
\providecommand{\n}{}
\providecommand{\o}{}
\providecommand{\p}{}
\providecommand{\q}{}
\providecommand{\r}{}
\providecommand{\s}{}
\providecommand{\t}{}
\providecommand{\u}{}
\providecommand{\v}{}
\providecommand{\w}{}
\providecommand{\x}{}
\providecommand{\y}{}
\providecommand{\z}{}
% Now can just use renewcommand

%%%%%%%%%%%%%%%%%%%%%%%%%%%%%%%%%%%%%%%%%%%%%%%%%%%%%%%%%%%%%%%%%%%%%%%%%%%%%%
% Tables
%%%%%%%%%%%%%%%%%%%%%%%%%%%%%%%%%%%%%%%%%%%%%%%%%%%%%%%%%%%%%%%%%%%%%%%%%%%%%%
% \begin{table}
%   \centering
%   \arraystretch{1.2}
%   \setlength{\tabcolsep}{5.5pt}
%   \rowcolors{2}{white}{lightbluishgrey}
%   \begin{tabular}{l r r r r r r r r r r r r r r r r r r}
%     \rowcolor{white}
%       col1 & \makecell{long name \\ col2} &  \\
%     \midrule
%       item1 & entry1\\
%     \bottomrule
%   \end{tabular}
%   \caption{ }
% \end{table}
\usepackage{tabularx}
\usepackage{booktabs}
\usepackage{makecell}
\usepackage[table]{xcolor}
\definecolor{white}{rgb}{1,1,1}
\definecolor{lightbluishgrey}{rgb}{0.76471,0.84824,0.91647}

%%%%%%%%%%%%%%%%%%%%%%%%%%%%%%%%%%%%%%%%%%%%%%%%%%%%%%%%%%%%%%%%%%%%%%%%%%%%%%
% Figures
%%%%%%%%%%%%%%%%%%%%%%%%%%%%%%%%%%%%%%%%%%%%%%%%%%%%%%%%%%%%%%%%%%%%%%%%%%%%%%
\usepackage{subcaption}
\usepackage{wrapfig}
\usepackage{graphicx}

\usepackage{listings}
%%%%%%%%%%%%%%%%%%%%%%%%%%%%%%%%%%%%%%%%%%%%%%%%%%%%%%%%%%%%%%%%%%%%%%%%%%%%%%
% Line widths and font sizes
%%%%%%%%%%%%%%%%%%%%%%%%%%%%%%%%%%%%%%%%%%%%%%%%%%%%%%%%%%%%%%%%%%%%%%%%%%%%%%
\usepackage{layouts}
\makeatletter 
% \layoutdetails will insert a table listing the layout sizes and fonts in use
\newcommand{\layoutdetails}{%
\begin{tabular}{ll}
 \texttt{\textbackslash{textwidth}} & \printinunitsof{in}\prntlen{\textwidth} \\
\texttt{\textbackslash{linewidth}} & \printinunitsof{in}\prntlen{\linewidth} \\
Main text font &  \f@size pt \f@family \\
\sffamily \small Caption text font &  \sffamily \small \f@size pt \f@family \\
\end{tabular}%
}
\makeatother 


\begin{document}

\title{Gender in SIGGRAPH Technical Papers, 2015-2021}

\author{Silvia Sell\'{a}n} \affiliation{\institution{University of Toronto}}

\begin{abstract}
\end{abstract}


\maketitle

\section{Methodology}

We reviewed all technical papers published on ACM Transactions on Graphics and
presented at either of the North American or Asian editions of SIGGRAPH between
January 2015 and December 2021, and recorded all of those that contained
explicit mentions of gender.

For each paper containing an explicit mention of gender, we annotated the type
of variable used to represent it ("binary", "non-binary" or "unclear"), its
definition given or implied in the paper (e.g., self-identification, voice
pitch, body proportions) and its use or uses in the paper (whether as a
parameter in the main proposed algorithm, as demographic information of user
study participants, etc.).

\section{Observations}

We compiled a total of 63 technical papers that mentioned gender explicitly. Our
main observations are recorded below:
\begin{enumerate}
    \item 45 papers (71\%) represented gender as a binary variable, and 18
    (29\%) do not specify the type of representation used. None of the reviewed
    papers represented gender explicitly as a non-binary variable.
    \item 20 papers mention gender as a parameter in an algorithm, 19 papers (30\%) mentioned gender as demographic information reported
    about their user study participants, 19 (30\%) did so as demographic composition of their dataset and 8 as demographic information about their test set.
    \item While no paper explictly defines their understanding of gender, some implicitly use it to refer to body 
\end{enumerate}

\end{document}