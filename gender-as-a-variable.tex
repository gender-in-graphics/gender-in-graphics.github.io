\PassOptionsToPackage{table}{xcolor}
\documentclass[acmtog,nonacm,review,balance=false]{acmart}
\citestyle{acmauthoryear}
\setcitestyle{square}

%%%%%%%%%%%%%%%%%%%%%%%%%%%%%%%%%%%%%%%%%%%%%%%%%%%%%%%%%%%%%%%%%%%%%%%%%%%%%%
% UNICODE: make θ
\usepackage[mathletters]{ucs}
\usepackage[utf8x]{inputenc}
%%%%%%%%%%%%%%%%%%%%%%%%%%%%%%%%%%%%%%%%%%%%%%%%%%%%%%%%%%%%%%%%%%%%%%%%%%%%%%

%%%%%%%%%%%%%%%%%%%%%%%%%%%%%%%%%%%%%%%%%%%%%%%%%%%%%%%%%%%%%%%%%%%%%%%%%%%%%%
% Referencing
%%%%%%%%%%%%%%%%%%%%%%%%%%%%%%%%%%%%%%%%%%%%%%%%%%%%%%%%%%%%%%%%%%%%%%%%%%%%%%
\newcommand{\refequ}[1]{Eq.~(\ref{equ:#1})}
\newcommand{\refapp}[1]{Appendix \ref{app:#1}}
\newcommand{\reffig}[1]{Figure~\ref{fig:#1}}
\newcommand{\refsec}[1]{Section~\ref{sec:#1}}
\newcommand{\silvia}[1]{{\bf\textcolor[rgb]{0.9,0.6,0.6}{Silvia: #1}}}
%%%%%%%%%%%%%%%%%%%%%%%%%%%%%%%%%%%%%%%%%%%%%%%%%%%%%%%%%%%%%%%%%%%%%%%%%%%%%%
% Characters
%%%%%%%%%%%%%%%%%%%%%%%%%%%%%%%%%%%%%%%%%%%%%%%%%%%%%%%%%%%%%%%%%%%%%%%%%%%%%%
% "Remove" any exist single-letter commands
\providecommand{\A}{}
\providecommand{\B}{}
\providecommand{\C}{}
\providecommand{\D}{}
\providecommand{\E}{}
\providecommand{\F}{}
\providecommand{\G}{}
\providecommand{\H}{}
\providecommand{\I}{}
\providecommand{\J}{}
\providecommand{\K}{}
\providecommand{\L}{}
\providecommand{\M}{}
\providecommand{\N}{}
\providecommand{\O}{}
\providecommand{\P}{}
\providecommand{\Q}{}
\providecommand{\R}{}
\providecommand{\S}{}
\providecommand{\T}{}
\providecommand{\U}{}
\providecommand{\V}{}
\providecommand{\W}{}
\providecommand{\X}{}
\providecommand{\Y}{}
\providecommand{\Z}{}
\providecommand{\a}{}
\providecommand{\b}{}
\providecommand{\c}{}
\providecommand{\d}{}
\providecommand{\e}{}
\providecommand{\f}{}
\providecommand{\g}{}
\providecommand{\h}{}
\providecommand{\i}{}
\providecommand{\j}{}
\providecommand{\k}{}
\providecommand{\l}{}
\providecommand{\m}{}
\providecommand{\n}{}
\providecommand{\o}{}
\providecommand{\p}{}
\providecommand{\q}{}
\providecommand{\r}{}
\providecommand{\s}{}
\providecommand{\t}{}
\providecommand{\u}{}
\providecommand{\v}{}
\providecommand{\w}{}
\providecommand{\x}{}
\providecommand{\y}{}
\providecommand{\z}{}
% Now can just use renewcommand

%%%%%%%%%%%%%%%%%%%%%%%%%%%%%%%%%%%%%%%%%%%%%%%%%%%%%%%%%%%%%%%%%%%%%%%%%%%%%%
% Tables
%%%%%%%%%%%%%%%%%%%%%%%%%%%%%%%%%%%%%%%%%%%%%%%%%%%%%%%%%%%%%%%%%%%%%%%%%%%%%%
% \begin{table}
%   \centering
%   \arraystretch{1.2}
%   \setlength{\tabcolsep}{5.5pt}
%   \rowcolors{2}{white}{lightbluishgrey}
%   \begin{tabular}{l r r r r r r r r r r r r r r r r r r}
%     \rowcolor{white}
%       col1 & \makecell{long name \\ col2} &  \\
%     \midrule
%       item1 & entry1\\
%     \bottomrule
%   \end{tabular}
%   \caption{ }
% \end{table}
\usepackage{tabularx}
\usepackage{booktabs}
\usepackage{makecell}
\usepackage[table]{xcolor}
\definecolor{white}{rgb}{1,1,1}
\definecolor{lightbluishgrey}{rgb}{0.76471,0.84824,0.91647}

%%%%%%%%%%%%%%%%%%%%%%%%%%%%%%%%%%%%%%%%%%%%%%%%%%%%%%%%%%%%%%%%%%%%%%%%%%%%%%
% Figures
%%%%%%%%%%%%%%%%%%%%%%%%%%%%%%%%%%%%%%%%%%%%%%%%%%%%%%%%%%%%%%%%%%%%%%%%%%%%%%
\usepackage{subcaption}
\usepackage{wrapfig}
\usepackage{graphicx}

\usepackage{listings}
%%%%%%%%%%%%%%%%%%%%%%%%%%%%%%%%%%%%%%%%%%%%%%%%%%%%%%%%%%%%%%%%%%%%%%%%%%%%%%
% Line widths and font sizes
%%%%%%%%%%%%%%%%%%%%%%%%%%%%%%%%%%%%%%%%%%%%%%%%%%%%%%%%%%%%%%%%%%%%%%%%%%%%%%
\usepackage{layouts}
\makeatletter 
% \layoutdetails will insert a table listing the layout sizes and fonts in use
\newcommand{\layoutdetails}{%
\begin{tabular}{ll}
 \texttt{\textbackslash{textwidth}} & \printinunitsof{in}\prntlen{\textwidth} \\
\texttt{\textbackslash{linewidth}} & \printinunitsof{in}\prntlen{\linewidth} \\
Main text font &  \f@size pt \f@family \\
\sffamily \small Caption text font &  \sffamily \small \f@size pt \f@family \\
\end{tabular}%
}
\makeatother 


\begin{document}

\title{On the use of gender as a variable in Computer Graphics research}

\author{Silvia Sell\'{a}n} \affiliation{\institution{University of Toronto}}

\begin{abstract}
\end{abstract}


\maketitle

\cite{phillips2020gamer}

% A person's self-identity is, by definition, not subject to debate, be it
% political or scientific. Similarly, it is not my intention to open or contribute
% to a ``debate'' on a person's individual right to a life in liberty, dignity and
% equality in rights with their fellow humans as stated in Article 1 of the United
% Nations' Universal Declaration of Human Rights.  

My focus is on examining the current use of a specific demographic
variable (gender) in the context of Computer Graphics research. I will argue
that (1) its use is a product of unscientific assumptions; (2) it is not
currently rigorously defined and thus its use is harmful to research
replicability; (3) its current use excludes a significant subset of the human
population and contributes to their ongoing prosecution, discrimination and
opression, thus going against ACM SIGGRAPH's Vision of \emph{enabling everyone
to tell their stories}; and (4) its current use can be easily modified to not
present problems (1)-(3).

\section{An unscientific assumption}

In all of the published Computer Graphics literature I've reviewed, ``gender''
is often used as a variable and, when it is, it is treated as a binary proxy for
a set of physical characteristics. The assumptions behind its use as a relevant
variable can and should be scientifically assessed.

The implicit, be it concious or unconcious, assumption in the literature seems
to be that, physically, every human's appearance or behavior can be modeled as
belonging to a very high-dimensional vector space $\mathcal{V}$ that could
theoretically be spanned by a very large set of (not orthogonal) basis vectors
$\{v_1,\dots, v_n\}$ spanning subspaces like weight, skin tone, eye color, hair
style, clothes worn, facial expression, etc. Furthermore, the assumption says,
the distribution of humans in $\mathcal{V}$ follows a bimodal law, and one of
the modes can be arbitrarily called female or woman while the other may be
called male or man. 

The above assumption is wrong, both in theory and in practice. Some of these
attributes, like height, broadness of shoulders, voice pitch or hair length, may
be empirically shown to \emph{independently} follow what is well approximated by
a bimodal distribution. What \emph{doesn't} follow theoretically from this fact
is that their \emph{joint probability} also follows a bimodal distribution: in
general, the joint probability of $k$ bimodal distributions can only be said to
be at most $2^k$-modal. This undeniably manifests in practice too: [example??]. Similar arguments can
be made with chest size, hip width, body-mass index, facial hair, hair style,
voice pitch, etc.

\section{A non rigurously defined variable}

The wrong bimodality assumption exposed above makes it so that researchers
usually make arbitrary decisions on what ``male'' and ``female'' mean. While the
specific definition used is not found on any paper I have surveyed, it is
probably safe to assume that researchers usually use these binary categories to
represent the modes of whichever physical attribute they subjectively consider
most important for their application.

In some cases, these attributes may be trivial to identify. For example, [cite
the thing] appears to use gender as a proxy for the shape of a person's torso to
aid with sensor placement. In other cases, like where gender is an input
variable to a neural network which learned on annotated data, the authors
themselves may not know exactly which human characteristics are being mapped to
this binary classification.

In most cases, the use of gender without a proper definition creates a model
ambiguity which is detrimental to the advancement of science. For example, a
voice recognition paper may refer to humans with higher pitched voices as female
and humans with lower pitched voices as male, or it may be may be discriminating
between certain speech patterns and word choices. Similarly, a full-body motion
capture algorithm may use gender as a proxy for certain unspecified
physiological attributes that aid with sensor positioning, for different ranges
of Body Mass Index or for which poses and gestures are traditionally adscribed
to a gender in a certain culture. Finally, a facial reconstruction algorithm may
claim to take a person's gender as input, without clarifying whether it refers
to the presence of facial hair or makeup, a larger or smaller jaw, or a person's
facial expression.

\section{The real-world impact}

As computer graphics researchers, we must be aware of the effect that our
arbitrary modelling decisions have in the real world. Our algorithms are
commercialized and used by governments, police departments, airlines and many
more private companies with which the public is increasingly forced to interact
in today's world.

A researcher's wrong assumption that a person's physiological characteristics
are bimodal instead of multimodal lead to TSA body-scanners that routinely
subject gender non-conforming people to ridicule and discrimination. A
researcher's ambiguity regarding the gender output of a recognition algorithm
makes it so gender non-conforming individuals can be often flagged for fraud if
their recognized gender does not match their government identification. A
medical imaging tool's use of gender with an unclear definition leads to
transgender individuals going undiagnosed with severe illnesses. Current trends
in the current use of gender as a variable in Computer Graphics research
directly contribute to further harm communities which are already at higher
risks of being mistreated by authorities and denied jobs or healthcare. 

Furthermore, graphics researcher must consider our role in shaping the whose
stories get to be told and who gets to seem themself be represented in the
entertainment culture. In one fell swoop, by assuming a wrong bimodality in
gender and making character models and animation rigs that use gender as a
binary variable, we exclude gender non-conforming individuals from every
videogame and movie created using our tool. By not examining our decisions
around the use of gender in our algorithms, we contribute to further
invisibilize already-invisible and marginalized communities, and in doing so
contribute also to their ongoing opression.

Finally, the current use of gender in the Computer Graphics literature creates a
hostile environment for gender non-conforming members of our research community,
which goes against [cite some siggraph inclusion goal]: by seeing colleagues and
collaborators consistently exclude us from their own research work, we are
reminded that we do not belong in this research community, encouraging us to
look for jobs elsewhere.


\section{Solutions}

There are generally four cases in which gender shows up as a variable in a
Computer Graphics research work, and our proposed best practices vary for each:

\begin{enumerate}
  \item As a proxy for a mode in the distribution of a single human
    characteristic; for example, to describe hair styles. In this case, we
    recommend the gender reference be substituted, in the interest of
    disambiguation, for the actual characteristic: for example, instead of
    ``male and female hairstyles'', say ``longer and shorter hairstyles''.
  \item As a proxy for many simultaneous modes in the distribution of human
    characteristics; for example, animation of ``male bodies'' or ``female
    bodies''. In this case, the researchers should examine the necessity of this
    variable, and should be aware of the diversity of modes of human existence
    they are excluding by making that design choice, as well as their potential
    societal implications. Furthermore, if an algorithm only accepts as input or
    output gender conforming types, this should be stated as a major
    scientifical limitation of their work.
  \item As the central object of study; for example, works that purport to
    ``detect'' gender or ``change'' a given model's gender as their main
    contribution. The potential harms of this literature are so vast that we
    recommend authors to contact all the societal stakeholders, including trans
    and gender non-conforming scientists and ethicists, on the appropriateness
    and relevance of the proposed work. 
  \item As aggregate demographic data on the participants on user studies, to
    guard against \emph{reference man}-type research outcomes. In the vast
    majority of cases, this should be self-reported by user participants and not
    assumed by the person conducting the study.
\end{enumerate}

\section{Open questions}

To-do

\bibliographystyle{ACM-Reference-Format}
\bibliography{references.bib}

\end{document}
